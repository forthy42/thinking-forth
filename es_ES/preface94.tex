\chapter*{Pr�logo a la edici�n de 1994}\label{preface94}
\pagestyle{headings}

\initial El \Forth{} Interest Group me honra en reimprimir
\emph{Pensando en \Forth{}}. Es grato saber que el libro puede haber
tenido valor para los simpatizantes y usuarios de \Forth{}.

Esta edici�n es una reproducci�n de la original, con tan s�lo
correcciones tipogr�ficas menores. Muchas cosas han ocurrido en los 10
a�os desde la publicaci�n original del libro, dejando obsoletas
algunas opiniones, o como mucho, arcaicas. Una ``edici�n corregida y
aumentada'' hubiera implicado reescribir muchas secciones, un esfuerzo
mayor del que puedo permitirme en este momento.

De todas las opiniones del libro, la que m�s lamento ver impresa es mi
cr�tica de la programaci�n orientadas a objetos. Desde que escrib�
este libro, he tenido el placer de escribir una aplicaci�n en una
versi�n de \Forth{} con apoyo para la programaci�n orientada a
objetos, desarrollada por Digalog Corp. de Ventura, California. No soy
un experto, pero est� claro que esa metodolog�a tiene mucho que
ofrecer.

Con todo esto, creo que muchas de las ideas de \emph{Pensando en
  \Forth{}} son tan v�lidas hoy en d�a como lo eran entonces.
Ciertamente, los comentarios de \person{Charles Moore} siguen siendo
una inspiraci�n viviente sobre la filosof�a que inici� el desarrollo
de \Forth{}.

Quiero agradecer a \person{Marlin Ouverson} su excelente trabajo,
luchando con paciencia contra los formatos de ficheros incompatibles y
los errorres del OCR, por llevar esta reimpresi�n a la vida.
